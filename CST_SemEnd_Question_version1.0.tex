%% This is a template of Semester -End question paper of College of Science and Technology, Royal University of Bhutan
%% It Uses Exam envinornment, However, Marks to be entered Manually just before a question. 
%% \totalpoints can be used after a question to validate the total marks. For this marks needs to be entered for all the parts.

%% : Version 1.0, Date: May 1, 2021

%%  Original Author : Ms. Tshering Denka, Modified by Kirtan Adhikari (adhikari.cst@rub.edu.bt)

%% Use \printanswers to print the answer. 

\documentclass[12pt, addpoints,a5paper]{exam}
\usepackage{graphicx}
\usepackage[normalem]{ulem}
\renewcommand\ULthickness{1.0pt}   %%---> For changing thickness of underline
\setlength\ULdepth{1.3ex}%\maxdimen ---> For changing depth of underline
%\printanswers %% --> If Answer is required to be displayed along with questions
\bracketedpoints
\pointsdroppedatright

\renewcommand{\questionlabel}{\textbf{Question No.}~\thequestion.}
\renewcommand\partlabel{\thequestion.\arabic{partno}}

%----------------------------------------------------------------------------------------
\usepackage{amsmath}
\usepackage{amssymb}
\usepackage{graphicx}
\usepackage[top=.65 in, bottom=1.25 in, left=0.6 in, right=0.6 in]{geometry}
\usepackage{lastpage}
\usepackage{times}
\usepackage{calc}
\usepackage{tikz}
\usetikzlibrary{decorations.markings}
\usetikzlibrary{arrows}
\usepackage{blkarray}

\tikzstyle{vertex}=[circle, draw, inner sep=0pt, minimum size=4.25pt]
\newcommand{\vertex}{\node[vertex]}
\newcounter{Angle}


%\pagestyle{fancy}
%\fancyhf{}
\cfoot{\footnotesize Page {\bfseries \thepage}\ of {\bfseries \pageref{LastPage}}}

%\renewcommand{\headrulewidth}{0pt}

\setlength{\parskip}{1.3ex plus 0.2ex minus 0.2ex}
\setlength{\parindent}{0pt}
%-----------------------------------------------------------------------------------------------------
\begin{document}
	\qquad \qquad \qquad \qquad\qquad \qquad\qquad {\bfseries $\boxed {\textbf{Student No.\,:} \qquad  \qquad \qquad}\,\,$}
	\bigskip 
	\noindent
	
	
	\begin{minipage}[c]{\textwidth}
		\begin{center}
			{\bfseries COLLEGE OF SCIENCE AND TECHNOLOGY\\
				ROYAL UNIVERSITY OF BHUTAN\\
				PHUENTSHOLING: BHUTAN}
		\end{center}
	\end{minipage}
	\bigskip
	
	\begin{center}
		{\bfseries SPRING SEMESTER EXAMINATION: 2021}
	\end{center}
	
	\begin{tabular}{lll}
		{\bfseries Class} & {\bfseries :} & {\bfseries B.E. Second Year Civil}\\[2ex]
		{\bfseries Module} & {\bfseries :} & {\bfseries Hydraulics}\\[2ex]
		{\bfseries Module Code} & {\bfseries :} & {\bfseries FMH202}\\[2ex]
		{\bfseries Max. Marks} & {\bfseries :} & {\bfseries 50}\\[2ex]
		{\bfseries Max. Time} & {\bfseries :} & {\bfseries 3 Hrs.}
	\end{tabular}
	
	\bigskip
	
	\hspace*{0pt}{\bfseries \, General Instructions:}
	\begin{enumerate}
		\item[{\bfseries 1.}] {\bfseries \itshape All the questions are compulsory from Section I}
		\item[{\bfseries 2.}] {\bfseries \itshape Answer any Four questions from Section II.}
		\item[{\bfseries 3.}] {\bfseries \itshape Sketch the figures neatly wherever necessary.}
		\item[{\bfseries 4.}] {\bfseries \itshape Usage of pencil apart from figures will be treated as rough work.}
		\item[{\bfseries 5.}] {\bfseries \itshape Assume any missing value suitably and mention on the ANSWER sheet}
		\item[{\bfseries 6.}] {\bfseries \itshape Do NOT write anything on this question paper.}
	\end{enumerate}
	
	
\newpage
	
%---------------------------------------------------------------------------------------------------------------
%                       								Section 1
%------------------------------------------------------------------------------------------------------------- 
	\centering \section*{Section I}
	\begin{center}
		{\small \bfseries Answer All The Questions }
	\end{center}	
	
%%------------------------------------- QUESTIONS--------------------------------------------------------------

	\begin{questions}
		
		\question {\bfseries } \hfill \textbf{[$\mathbf{2\times 10 = 20 }$]}% Starting Point : Type the question hereafter
		\ 
		
		\begin{parts}
		
		\part[5] 
		For a flow of water in a rectangular channel 0f 4.0 m width and depth of flow of 2.25 m, the \textit{Darcy-Weisbach} frictional factor \textit{f} is estimated to be equal to 0.02. Estimate the value of \textit{Chezy's} coefficient and \textit{Manning's} coefficient?
		%-------------------------------
		\begin{choices} % Provide the choices
			\choice C=59.6, n = 0.0163
			\choice C=60.6, n = 0.0163
			\choice C=61.6, n = 0.0163
			\CorrectChoice C=62.6, n = 0.0163 % the correct answer
		\end{choices}
		%-----------------------------
		
		%----------------------------
		\begin{solution} % Type your solution here
			The relation-ship among these coefficient are \\
			$$C = \sqrt{\frac{8g}{f}} = \frac{1}{n}R^{1/6}$$
			$$Area = By = 4* 2.5 = 10 m^2$$
			$$P = B + 2y = 4 + (2*2.5) = 9m$$
			$$R = \frac{A}{P} = 1.11m$$
			$$ C = \sqrt{\frac{8g}{f}} = \sqrt{\frac{8g}{0.02}} = 62.6$$
			$$n = \frac{1}{C}R^{1/6} = \frac{1}{62.6}1.111^{1/6} = 0.0163$$
		\end{solution} % End of a Question and Solution
		%----------------------------
		\part[5] 
		For a flow of water in a rectangular channel 0f 4.0 m width and depth of flow of 2.25 m, the \textit{Darcy-Weisbach} frictional factor \textit{f} is estimated to be equal to 0.02. Estimate the value of \textit{Chezy's} coefficient and \textit{Manning's} coefficient?
		%-------------------------------
		\begin{choices} % Provide the choices
			\choice C=59.6, n = 0.0163
			\choice C=60.6, n = 0.0163
			\choice C=61.6, n = 0.0163
			\CorrectChoice C=62.6, n = 0.0163 % the correct answer
		\end{choices}
		%-----------------------------
		
		%----------------------------
		\begin{solution} % Type your solution here
			The relation-ship among these coefficient are \\
			$$C = \sqrt{\frac{8g}{f}} = \frac{1}{n}R^{1/6}$$
			$$Area = By = 4* 2.5 = 10 m^2$$
			$$P = B + 2y = 4 + (2*2.5) = 9m$$
			$$R = \frac{A}{P} = 1.11m$$
			$$ C = \sqrt{\frac{8g}{f}} = \sqrt{\frac{8g}{0.02}} = 62.6$$
			$$n = \frac{1}{C}R^{1/6} = \frac{1}{62.6}1.111^{1/6} = 0.0163$$
		\end{solution} % End of a Question and Solution
		%----------------------------
		\end{parts}
		
	
	
%---------------------------------------------------------------------------------------------------------------
%                       								Section 2
%------------------------------------------------------------------------------------------------------------- 
\newpage
\centering \section*{Section II}

	\begin{center}

	\vspace{0.15in}
	{\small \bfseries Solve \textbf{ANY FOUR Questions} }

	\end{center}	

	
%%--------------------------------------------Questions------------------------------------------------------
%{\bfseries Question No. 1} \hfill \textbf{[$\mathbf{10\times 1 = 10}$]}		
%%%%%%%%%%%%%%%%%%%%%%%%%%%%%%%%%%%%%%%%%%%%%%%%%%%%%%%%%%%%%%%%%%%%%%%%%%%%%%%%%%%%%%%	
%\setcounter{question}{1}
	
		
		\question {\bfseries } \hfill \textbf{[$\mathbf{5,2,3 }$]}
		\
		\begin{parts}
			\part[5] Type question 2.1 here. Type question 2.2 hereType question 2 hereType question 2 hereType question 2 hereType question 2 hereType question 2 hereType question 2 hereType question 2 hereType question 2 hereType question 2 hereType question 2 here %\droppoints   
			\vspace{0.05in} 
			\begin{solution}
				Type the solution for Question 2
				
			\end{solution}		
		
		\part[5] Type question 2.2 here. Type question 2.2 here.Type question 3 here.Type question 3 here.Type question 3 here.Type question 3 here.Type question 3 here.Type question 3 here.Type question 3 here.Type question 3 here.Type question 3 here.Type question 3 here.Type question 3 here. %\droppoints   
		\vspace{0.05in} 
		
		\begin{solution}
			Type the solution for Question 2
			
		\end{solution}
	
    \end{parts}

	\question {\bfseries } \hfill \textbf{[$\mathbf{5,2,3 }$]}
\
\begin{parts}
	
	\part[5] Type question 3.1 here. Type question 2.2 hereType question 2 hereType question 2 hereType question 2 hereType question 2 hereType question 2 hereType question 2 hereType question 2 hereType question 2 hereType question 2 hereType question 2 here %\droppoints   
	\vspace{0.05in} 
	
	\begin{solution}
		Type the solution for Question 2
		
	\end{solution}
	
	\part[5] Type question 3.2 here. Type question 2.3 here.Type question 3 here.Type question 3 here.Type question 3 here.Type question 3 here.Type question 3 here.Type question 3 here.Type question 3 here.Type question 3 here.Type question 3 here.Type question 3 here.Type question 3 here. %\droppoints   
	\vspace{0.05in} 
	
	\begin{solution}
		Type the solution for Question 2
		
	\end{solution}
 

\end{parts}	
			
\end{questions}
	
\vspace{0.75in}	
\end{document}